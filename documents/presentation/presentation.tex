\documentclass{beamer}
\usepackage[utf8]{inputenc}
\setbeamertemplate{items}[ball] 
\title{R Users Group}
\subtitle{Intro to Docker - Why and How to Use It}
\author{David Thole}
\institute{University of Iowa, College of Pharmacy}
\date{2019/01/24}
\logo{\includegraphics[width=50px]{images/docker.png}}
\AtBeginSection[]
{
  \begin{frame}
    \frametitle{Agenda}
    \tableofcontents[currentsection]
  \end{frame}
}

\begin{document}
\frame{\titlepage}

\begin{frame}
    \frametitle{Agenda}
    \tableofcontents
\end{frame}

\section{1. What is it, how to start using it?}
\begin{frame}
    \frametitle{Purpose of Docker}
    Came about from the concept of "Depenency Hell", and "It works on my machine" which is the tightly coupled dependencies that are difficult to reproduce
    on other machines.
\end{frame}

\begin{frame}
    \frametitle{Why use Docker? (System Admin Perspective)}
    \begin{enumerate}
        \item Ability to run PowerShell/Bash on any OS.
        \item Ability to publish workflows that include the dependencies for that specific process, or set of processes.
        \item Ability to reproduce an environment in a deterministic manner (oops, forgot to install X)
    \end{enumerate}
\end{frame}

\begin{frame}
    \frametitle{Why use Docker? (Developer Perspective)}
    \begin{enumerate}
        \item Ability to bundle dependencies for a project into one bundle (No more needing version .NET 4.5 runtime installed on system X)
        \item Ability to have containers more accurate between test and production (same container can be used with different entrypoints)
        \item Ability to abstract dependencies of a development machine (No more install X, then Y, then Z - for each developer, spending a day setting up a new machine)
        \item Ability to test new software/products quickly
        \item Ability to script out deployment/testing much easier than current solutions
    \end{enumerate}
\end{frame}

\begin{frame}
    \frametitle{Why use Docker? (DBA/Analyst Perspective)}
    \begin{enumerate}
        \item Ability to test data sets easier (script the loading from a backup)
        \item Ability to create working sets (e.g. working on project X's data, now project Y)
    \end{enumerate}
\end{frame}

\begin{frame}
    \frametitle{Why use Docker? (Researcher Perspective)}
    \begin{enumerate}
        \item Ability to reproduce results using an exact setup (vs it ran on machine X 3 years ago, and has since been reformatted)
        \item Ability to script an experiment's processing, to enforce standardization, but to aid in reproducability
    \end{enumerate}
\end{frame}

\section{2. How it works}
\begin{frame}
    \frametitle{How it works}
    \includegraphics[width=290px]{images/HowItWorks.png}
\end{frame}

\section{3. Basic Usage}
\begin{frame}
    \frametitle{Basic Commands}
    \textcolor{red}{\textbf{Main Command}}: docker \\
    \textcolor{blue}{\textbf{Script Reference}}: setup.sh \\
    \textcolor{blue}{\textbf{Script Reference}}: basics.sh
\end{frame}

\begin{frame}
    \frametitle{Running Containers}
    \textcolor{red}{\textbf{Main Command}}: docker run \\
    \textcolor{blue}{\textbf{Script Reference (1)}}: interactive.sh \\
    \textcolor{blue}{\textbf{Script Reference (2)}}: daemon.sh
\end{frame}

\begin{frame}
    \frametitle{Building Containers}
    \textcolor{red}{\textbf{Main Command}}: docker build \\
    \textcolor{blue}{\textbf{Script Reference (1)}}: dotnet/buildme.sh Dockerfile \\
    \textcolor{blue}{\textbf{Script Reference (2)}}: documents/nbs\_facility/*
\end{frame}

\begin{frame}
    \frametitle{Using Multiple Containers}
    \textcolor{red}{\textbf{Main Command}}: docker run \\
    \textcolor{blue}{\textbf{Script Reference (1)}}: multicontainers.sh
\end{frame}

\section{Closing}
\begin{frame}
    \frametitle{4. In Closing}
    Resources:
    \begin{itemize}
        \item Pluralsight (Docker Path)
        \item Docker Up \& Running (O'Reilly)
        \item Dockerhub (best place to look at what's out there)
    \end{itemize}
    Useful/Fun Containers:
    \begin{itemize}
        \item thedarktrumpet/rstudiojava - Studio, Shiny (For R Use)
        \item microsoft/mssql-server-linux - Microsoft SQL Server
        \item jupyter/all-spark-notebook
    \end{itemize}
\end{frame}
\end{document}